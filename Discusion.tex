\chapter{Discusión}
\label{discu}

Es importante abrir una discutir sobre los aspectos positivos y negativos del trabajo de tesis y las aportaciones que han hecho a la ciencia, entre otros.

Siendo el aporte principal de este trabajo un modelo computacional en el área de la optimización, no se evidencias connotaciones éticas ni aspectos que afecten directamente a la humanidad, pero como todos los proyectos que hacen parte de la Inteligencia Artificial, podrá ser fuente de discusión sobre su uso adecuado en el futuro.

Este proyecto, como otros diseñados para apoyar la toma de decisiones, busca, básicamente hacer predicciones sobre lo que se espera que ocurra en determinado contexto; y, es la predicción uno de los aspectos que la Inteligencia Artificial ha conseguido ahondar, con resultados impresionantes para el años 2021, en que nos encontramos.

Tales resultados podrán aportar positivamente a la humanidad, como es el caso de la aplicabilidad que se prevé para este trabajo en el apoyo a casos, siendo los del área de la medicina los casos más frecuentes, por su incertidumbre y baja repetitividad. 

Pero queda abierta la posibilidad, como ha sucedido con diferentes avances científicos, de que los resultados se adapten para beneficios egoístas o para ideas maliciosas que en este momento no se puedan visualizar.

Ahora bien, la tesis aquí expuesta pretende hacer una porte positivo a la ciencia y espera que tenga provecho para la humanidad.

