\chapter{Conclusiones y trabajo futuro}

Al finalizar el trabajo de tesis, se establece que se ha dado solución a lo planteado en la sistematización del problema científico, y por ende, en los objetivos propuestos, así:

\begin{itemize}
    \item Se logró con el modelo propuesto, dar manejo a la incertidumbre en la toma de decisiones para problemas que se puedan representar con el grafo por etapas propuesto, desde una base matemática y teórica.    
    \item Se construyó un modelo de aprendizaje y se probó la convergencia de la ganancia hallada hacia una ganancia real, con valores aleatorios para el grafo propuesto.   
    \item El modelo de aprendizaje maneja como parámetros la cantidad de tiempos de ejecución y el valor de un factor de aprendizaje $\gamma$, que dosifica la magnitud de la ganancia que se ha de tener en cuenta como retroalimentación en el aprendizaje en cada iteración.
    \item Se encuentra que el uso de las teorías de los \textit{n-armed-bandits}, facilitan el desarrollo de una aplicación computacional sencilla, como se mostró en los cuadros de algoritmos del documento.   
    \item Se logra modelar, en el grafo por etapas, el manejo de casos dinámicos por parte de las BPMS, partiendo de la notación CMMN y siguiendo las recomendaciones consignadas en el capítulo \ref{aplica}.   
\end{itemize} 

El modelo propuesto es una implementación, hasta donde sabemos, nueva y muy simple en el campo de aprendizaje por refuerzo usando probabilidades. Se inscribe dentro del campo de los sistemas de decisión de Markov de estado finito y tiempo discreto (por episodios) con un número de tiempos en cada episodio también finito, específicamente para problemas dispuestos en un grafo por etapas, donde se requiere aprender de forma eficiente una trayectoria óptima que debe seguir un agente.

La estructura del modelo de decisión como grafo dirigido por etapas permite construir un algoritmo en el que las probabilidades de transición entre estados se actualizan directamente a partir de la función de valor estimada para cada nodo, dado que la estructura del grafo permite un cálculo eficiente de la constante de normalización de dichas probabilidades.

Aunque la aproximación de las probabilidades de transición de entre nodos, generó resultados satisfactorios, es un trabajo en el que se puede avanzar en varios aspectos:
\begin{itemize}
    \item Modificar la aplicación para que pueda dar el resultado, de forma confiable y con un menor número de iteraciones.    
    \item Automatizar la adaptación de parámetros para grafos con un alto número de etapas.  \item Comprobación de la eficacia del modelo variando la densidad del grafo.    
    \item Permitir la no selección de un nodo en alguna de las etapas.
    \item Permitir que la selección de un nodo en alguna de las etapas, esté apoyada por reglas o condiciones.    
    \item Modelar un problema de planificación automática con el grafo por etapas, adaptándolo desde las características propias del grafo de tipo And/Or.   
\end{itemize} 

