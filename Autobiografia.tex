%Autobiografia

\chapter*{Resumen autobiográfico}
\thispagestyle{empty}

\begin{center}
\autor

Candidato para obtener el grado de\\
\grado\\
\orientacion\bigskip

\uanl\\
\fime\bigskip

Tesis:\\
\textsc{\large\titulo}
\end{center}\bigskip

%Aquí va tu historia
Nacida en la  ciudad de Tunja, capital del departamento de Boyacá en Colombia. Hija de Laurentino Uribe López (qepd) y de María Antonia Sandoval Garavito. Cursó estudios de primaria en la Normal femenina de Tunja y en el Instituto Integrado Guillermo León Valencia de Duitama, ciudad cercana a Tunja, donde culminó su bachillerato en el Colegio Salesiano. 

Sus formación profesional de pregrado fue como Ingeniera de Sistemas en la Universidad Industrial de Santander, en la ciudad de Bucaramanga (Colombia) y de maestría fue como Magister en Ingeniería de Sistemas de la Universidad Nacional de Colombia, en Bogotá (Colombia).

Desde su graduación como profesional, se ha desempeñado como docente de la Universidad de Boyacá en su tierra natal, donde ha sido representante docente ante los consejos de facultad y académico, Secretaria del Consejo de Facultad de Ciencias e Ingeniería y Directora del programa de Ingeniería de Sistemas en la misma universidad. También se ha desempeñó como docente ocasional en universidades de Bogotá como la Universidad Nacional de Colombia, la Universidad Autónoma, la Universidad Católica de Colombia, la Universidad América, la Universidad Javeriana y la Universidad Agraria.

Su gusto por la Matemática le ha permitido orientar algunas de las asignaturas de matemática para ingeniería, como lógica, matemática discreta, teoría de la computación, métodos numéricos e investigación operacional, así como la matemática básica, que buscar nivelar los conocimientos que los estudiantes de primer semestre traen de sus diferentes centros educativos de secundaria, siendo esta su asignatura de mayor éxito, por el amor a la matemática que siembra en sus estudiantes.